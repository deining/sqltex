\documentclass{article}
\newcommand{\bs}{\ensuremath{\backslash}}
\newcommand{\vs}{\vspace{3mm}}
\usepackage{makeidx}
\begin{document}

\title{SQL\TeX\\v1.4}
\date{May 2, 2002}
\author{Oscar van Eijk}

\maketitle

\hrulefill

\tableofcontents

\hrulefill

\section{Introduction}

SQL\TeX\ is a preprocessor to enable the use of SQL statements in \LaTeX. It is a perl script that reads
an input file containing the SQL commands, and writes a \LaTeX\ file that can be processed with your
\LaTeX\ package.

The SQL commands will be replaced by their values. It's possible to select a single field for substitution
substitution in your \LaTeX\ document, or to be used as input in another SQL command.

When an SQL command returns multiple fields and or rows, the values can only be used for substitution
in the document.

\subsection{Known limitations}

\begin{itemize}
\item The current version only supports the MySQL database environment.
\item SQL\TeX\ reads only one input file; the \LaTeX\ \texttt{\bs include} directive is ignored
\item Currently, only 9 command- line parameters (1-9), and 10 variables (0-9) can be used in SQL
statements
\item With SQL\TeX\, it is only possible to read from a database (no updates).
\end{itemize}

\section{Installing SQL\TeX}

Before installing SQL\TeX, you need to have it. The latest version can always be found at
\textsf{http://software.oveas.net/sqltex}.
The download consists of this do\-cumentation, an installation script for Unix
(\texttt{install}), and the Perl script \texttt{SQLTeX}. The last file is all
you actually need.

\vs

On a Unix system, make sure the file \texttt{install} is executable by issueing
the command:\\
\texttt{bash\$ chmod +x install}\\
then execute it with:\\
\texttt{bash\$ ./install}

The script will ask in which directory SQL\TeX\ should be installed. If you are
logged in as `root', the default will be \texttt{/usr/local/bin}, otherwise the
current directory.\\
Make sure the directory where SQL\TeX\ is installed is in your path.

\vs

For other operating systems, there is no install script, you will have to install
it manually.

On \textsc{OpenVMS} it would be something like:\\
\texttt{\$ SET FILE/PROTECTION=(W:RE) SQLTEX.\footnote{Note the dot (`.') at the end of the file; on
\textsc{OpenVMS} systems, all files must to have a file extension, which can
be empty, in which case the filename ends with a dot.}\\
\$ COPY SQLTEX. SYS\$SYSTEM:}\\
However, on \textsc{OpenVMS} you also need to define the command \texttt{SQLTEX} by setting a symbol,
either in the \texttt{LOGIN.COM} for all users who need to execute this script, or in some group-- or
system wide login procedure, with the command:\\
\texttt{\$ SQLTEX :== "PERL SYS\$SYSTEM:SQLTEX."}\\

\subsection{Configuration}\label{config}

The program starts with a configuration section. The default values are displayed
here:

\vs

{\footnotesize\begin{verbatim}
#
################################################################################
# Configurable part
#
$main::texex       = "tex";    # default tex- file extension
$main::stx         = "_stx";   # file name extension to insert before the last '.'
#
$main::cmd_prefix  = "sql";    # prefix for sql-commands (\sql<command>[]{})
$main::sql_open    = "db";     # database declaration, e.g. \sqldb[user,passw]{database}
$main::sql_field   = "field";  # select a single field from db, e.g. \sqlfield{select field from....}
$main::sql_row     = "row";    # select rows from db, e.g. \sqlfield{select * from....}
#
$main::less_av     = 1;        # Is the command 'less' available on this system ?
$main::more_av     = 1;        # Is the command 'more' available on this system ?
#
################################################################################
# Do not make any modifications below this line                                #
################################################################################
\end{verbatim}}

\vs

These values are default values; most values can be overwritten using command line
options (see section~\ref{cmdline}). When the command line options are
omitted, the default values from the configuration section will be used.

\begin{description}

\item[\$main::texex] The default file extension for \LaTeX\ file. When SQL\TeX\ is called, the first
parameter should be the name of the input file. If this filename has no extension,
SQL\TeX\ looks for one with the default extension.

\item[\$main::stx] An output file can be given explicitly using the `\texttt{-o}' option. When omitted,
SQL\TeX\ composes an output file name using this string.\\
E.g, if your input file is called \texttt{db-doc.tex}, SQL\TeX\ will produce an
outputfile with the name \texttt{db-doc\_stx.tex}.

\item[\$main::cmd\_prefix]\label{prefix} SQL\TeX\ looks for SQL commands in the input file. Commands are specified in the
same way all \LaTeX\ commands are specified: a backslash (\bs) followed by the
name of the command.\\
All SQL\TeX\ commands start with the same string. By default, this is the string
\texttt{\textbf{sql}}. When user commands are defined that start with the same
string, this can be changed here to prevent conflicts.

\item[\$main::sql\_open] This string is appended to the \$main::cmd\_prefix to form the complete SQL\TeX\
for opening a database.\\
With the default configuration this command is ``\texttt{\bs sqldb}''.

\item[\$main::sql\_field] This string is appended to the \$main::cmd\_prefix to form the complete SQL\TeX\
to read a single field from the database.\\
With the default configuration this command is ``\texttt{\bs sqlfield}''.

\item[\$main::sql\_row] This string is appended to the \$main::cmd\_prefix to form the complete SQL\TeX\
to read one or more rows from the database.\\
With the default configuration this command is ``\texttt{\bs sqlrow}''.

\item[\$main::less\_av \& \$main::more\_av] These settings are used to determin how the \textsl{help} output should be displayed. If the
command `\texttt{less}' is available on the current system, the output will be
parsed through this program. Otherwise the output will be parsed through the
program `\texttt{more}' if availeble. Both programs are usually available on Unix
system (\texttt{more} is standard on most Unix systems), but ports for other
operating systems are available as well.\\
Set the values to ``0'' for the program(s) that is (are) not available, or if
you don't want to use it.\\
If none of these programs is available, the \textsl{help} output is plain echoed
to the display.

\end{description}

\section{Write your SQL\TeX\ file}

For SQL\TeX, you write your \LaTeX\ document just as you're used to. SQL\TeX\ provides you with
some extra commands that you can include in your file.\\
The basic format\footnote{in this document, in all examples will be asumed the default values in the
configuration section as described in section~\ref{config}, have not been changed} of an SQL\TeX\ command is: \\
\texttt{\bs sql\emph{cmd}[options]\{SQL statement\}}

\vs

All SQL\TeX\ commands can be specified anywhere in a line, and can span multiple lines.
When SQL\TeX\ executes, the commands are read, executed, and their results---if they return
any---are written to the output:

\vs

\begin{minipage}[t]{0.5\textwidth}\textsl{Input file:}\\\texttt{\footnotesize{\bs documentclass[article] \\
\bs pagestyle\{empty\} \\
\bs sqldb[oscar]\{mydb\} \\
\bs begin\{document\} \\
}}\end{minipage}\hfill\begin{minipage}[t]{0.5\textwidth}\textsl{Output file:}\\\texttt{\footnotesize{\bs documentclass[article] \\
\bs pagestyle\{empty\} \\
 \\
\bs begin\{document\} \\
}}\end{minipage}

\vs

Above you see the SQL\TeX\ command \texttt{\bs sqldb} was removed. Only the command was removed, not
the \textsl{newline} character at the end of the line, so an empty line will be printed instead.
The example below shows the output is an SQL\TeX\ command was found on a line with other \LaTeX\
directives:

\vs

\begin{minipage}[t]{0.5\textwidth}\textsl{Input file:}\\\texttt{\footnotesize{\bs documentclass[article] \\
\bs pagestyle\{empty\}\bs sqldb[oscar]\{mydb\} \\
\bs begin\{document\} \\
\hrulefill}}\end{minipage}\hfill\begin{minipage}[t]{0.5\textwidth}\textsl{Output file:}\\\texttt{\footnotesize{\bs documentclass[article] \\
\bs pagestyle\{empty\} \\
\bs begin\{document\} \\
}}\end{minipage}

\vs

In these examples the SQL\TeX\ commands did not return a value. When commands actually read from
the database, the returned value is written instead:

\vs

\begin{minipage}[t]{0.5\textwidth}\textsl{Input file:}\\\texttt{\footnotesize{This invoice has \bs sqlfield\{SELECT COUNT(*) FROM INVOICE\_LINE \\
WHERE INVOICE\_NR = 12345\} lines.\\
\hrulefill}}\end{minipage}\hfill\begin{minipage}[t]{0.5\textwidth}\textsl{Output file:}\\\texttt{\footnotesize{This invoice has 4 lines \\
}}\end{minipage}

\subsection{SQL statements}\label{sqlstatements}

This document assumes the reader is familiar with SQL commands. This section only tells something about
implementing them in SQL\TeX\ files, especially with the use of command parameters and variables.
Details about the SQL\TeX\ commands will be described in the next sections.

\vs

Let's look at a simple example. Suppose we want to retreive all header information from the database
for a specific invoice. The SQL statement could look something like this: \\
\texttt{SELECT $\ast$ FROM INVOICE WHERE INVOICE\_NR = 12345;}\\
To implement this statement in an SQL\TeX\ file, the \texttt{\bs sqlrow} command should be used (see
section~\ref{sqlrow}):

First, it is important to know that SQL statements should \textit{not} contain the ending semicolon (;) in
any of the SQL\TeX\ commands. The command in SQL\TeX\ would be:\\
\texttt{\bs sqlrow\{SELECT $\ast$ FROM INVOICE WHERE INVOICE\_NR = 12345\}}

Next, SQL\TeX\ would be useless if you have to change your input file every time you want to generate
the same document for another invoice.

\vs

Therefore, you parameters or variables can be used in your SQL statement. Parameters are given at the command
line (see section~\ref{params}), variables can be defined using the \texttt{\bs sqlfield} command as
described in section~\ref{vars}.

Given the example above, the invoice number can be passed as a parameter by rewriting the command as: \\
\texttt{\bs sqlrow\{SELECT $\ast$ FROM INVOICE WHERE INVOICE\_NR = \$PAR1\}} \\
or as as variable with the code line: \\
\texttt{\bs sqlrow\{SELECT $\ast$ FROM INVOICE WHERE INVOICE\_NR = \$VAR0\}}

Note you have to know what datatype is expected by your database. In the example here the datatype is
\textsc{integer}. If the field ``\textsc{invoice\_nr}'' contains a \textsc{varchar} type, the
\texttt{\$PAR}amater or \texttt{\$VAR}iable should be enclosed by quotes: \\
\texttt{\bs sqlrow\{SELECT $\ast$ FROM INVOICE WHERE INVOICE\_NR = '\$PAR1'\}}

\subsection{Opening the database}

Before any information can be read from a database, this database should be opened.
This is done with the \texttt{\textbf{\bs sqldb}} command.
\texttt{\bs sqldb} requires the name of the dabatase. Optionally, a username and password can be given. When
omitted, SQL\TeX\ assumes no username and password is required to connect to the database (the user that
executes SQL\TeX\ should have access to the specified database). \\
The format of the command is:\\
\texttt{\bs sqldb[username,password]\{database\}}

The command can be used anywhere in your input file, but should occur before the first command that tries to
read data from the database.

\subsection{Reading a single field}\label{sqlfield}

When a single field of information is to be read from the database, the command \texttt{\textbf{\bs sqlfield}}
is used. By default, the command in the inputfile is replaced by its result in the outputfile.\\
The SQL command is enclosed by curly braces. Square brackets can optionally be used to enter some extra options.
Currently, the only supported option is \texttt{setvar} (see section~\ref{vars}).

The full syntax or the \texttt{\bs sqlfield} command is:\\
\texttt{\bs sqlrow[\textit{options}]\{SELECT \textit{fieldname} FROM \textit{tablename} WHERE \textit{your where-clause}\}} \\
By default, the SQL\TeX\ command is replaced with the value returned by the SQL query. This behaviour
can be changed with options.


\subsubsection{Define variables}\label{vars}

The \texttt{\bs sqlfield} can also be used to set a variable. The value returned by the SQL query is not
displayed in this case. Instead, a variable is created which can be used in any other SQL query later in
the document (see also section~\ref{sqlstatements}).

Therefore, the option \texttt{\textbf{[setvar=\textit{n}]}} is used, where \textit{n} is an integer between
0 and 9.

\vs

Suppose you have an invoice in \LaTeX. SQL\TeX\ is executed to retrieve the invoice header information
from the database for a specific customer. Next, the invoice lines are read from the database.

You could pass the invoice number as a paramater to SQL\TeX\ for use in your queries, but that could
change every month. It is easier to :\\
\begin{itemize}
\item pass the customer number as a parameter,
\item retrieve the current date (asuming that is the invoice date as stored in
the database by another program), and store it in a variable: \\
\texttt{\bs sqlfield[setvar=0]\{SELECT DATE\_FORMAT(NOW(), "\%Y-\%m-\%d")\}} \\
This creates a variable that can be used as \texttt{\$VAR0},
\item retrieve the invoice number using the customer number (a command line parameter,
see also section~\ref{params}) and the variable containing the invoice date.
Store this invoice number in \texttt{\$VAR1}: \\
\texttt{\bs sqlfield[setvar=1]\{SELECT INVOICE\_NR FROM INVOICES \\
WHERE CUST\_NR = '\$PAR1' AND INVOICE\_DATE = '\$VAR0'\}}
\item use \texttt{\$VAR1} to retrieve all invoice information.
\end{itemize}

\vs

The SQL queries used here do not display any output in your \LaTeX document.


\subsection{Reading rows of data}\label{sqlrow}

When an SQL query returns more information than one single field, the SQL\TeX\
command \texttt{\textbf{\bs sqlrow}} should be used. As with the \texttt{\bs sqlfield},
command, SQL\TeX\ replaces the command with the values it returns, but \texttt{\bs sqlrow}
accepts different options for formating the output.

\vs

By default, fields are separated by a comma and a blank (`\texttt{,~}'), and rows by 
a newline character (`\texttt{\bs\bs}'). To change this, the options ``\texttt{fldsep}''
and ``\texttt{rowsep}'' can be used.

e.g. In a \texttt{tabular} enviroment the fields should be seperated by an amphesand (\texttt{\&}),
perhaps a line should seperate the rows of information. (\texttt{\bs\bs~\bs hline}).
To do this, the options can be used with \texttt{\bs sqlrow} as shown here: \\
\texttt{\bs sqlrow[fldsep=\&,rowsep=\bs\bs~\bs hline]\{SELECT I.LINE\_NR, A.ARTICLE\_NR, 
A.PRICE, I.AMOUNT, (A.PRICE * I.AMOUNT) FROM ARTICLE A, INVOICE\_LINE I WHERE I.INVOICE\_NR = \$VAR1 
AND I.ARTICLE\_NR = A.ARTICLE\_NR\}}

\vs

This will produce an output like: \\
\texttt{1 \& 9712 \& 12 \& 1 \& 12 \bs\bs~\bs hline \\
2 \& 4768 \& 9.75 \& 3 \& 29,25 \bs\bs~\bs hline \\
3 \& 4363 \& 1.95 \& 10 \& 19.5 \bs\bs~\bs hline \\
4 \& 8375 \& 12.5 \& 2 \& 25 \bs\bs~\bs hline}


\section{Process your SQL\TeX\ file}

To process you SQL\TeX\ file and create a \LaTeX\ file with all information read from
the database, call SQL\TeX\ with the parameter(s) and (optional) commandline options as
described here:

\subsection{Parameters}\label{params}

SQL\TeX\ accepts more than one parameter. The first parameter is required; this should
be the input file, pointing to your \LaTeX\ document containing the SQL\TeX\ commands.

By default, SQL\TeX\ looks for a file with extension `\texttt{.tex}'.

\vs

All other parameters are used by the queries, if required. If an SQL query contains the
string \texttt{\$PAR\textit{n}}\footnote{where \textit{n} is a number between 1 and 9. Note 
parameter `0' cannot be used, since that contains the filename!}, it is replaced by that parameter
(see also section~\ref{sqlstatements}).

\subsection{Command line options}\label{cmdline}

SQL\TeX\ accepts the followint command- line options:

\begin{description}
\item[-E \textit{string}] replace input file extension in outputfile:
\texttt{input.tex} will be \texttt{input.\textit{string}} \\
For further notes, see option \texttt{-e} below
\item[-N] \texttt{NULL} return values allowed. By default SQL\TeX\ exits if a
query returns an empty set.
\item[-P] prompt for database password. This overwrites the password in the
input file.
\item[-U \textit{user}] database username. This overwrites the username in the
input file.
\item[-V] print version number and exit.
\item[-e \textit{string}] add \textit{string} to the output filename:
\texttt{input.tex} will be \texttt{input\textit{string}.tex}. This overwrites
the configuration setting \texttt{\$main::stx} \\
In \textit{string}, the values between curly braces \{\} will be substituted:
\begin{description}
\item[P\textit{n}] parameter \textit{n}
\item[M] current monthname (\textit{Mon})
\item[W] current weekday (\textit{Wdy})
\item[D] current date (\textit{yyyymmdd})
\item[DT] current date and time (\textit{yyyymmddhhmmss})
\item[T] current time (\textit{hhmmss})
\end{description}
e.g., the command `\texttt{SQLTeX -e \_\{P1\}\_\{W\} my\_file code}'
will read `\texttt{my\_file.tex}' and write `\texttt{myfile\_code\_Tue.tex}'
The same command, but with option \texttt{-E} would create the
outputfile \texttt{myfile.\_code\_Tuesday}
By default (without \texttt{-e} or \texttt{-E}) the outputfile
\texttt{myfile\_stx.tex} would have been written.
The options \texttt{-E} and \texttt{-e} cannot be used together or with \texttt{-o}.
\item[-f] force overwrite of existing files. By default, SQL\TeX\ exists with a
warning message it the outputfile already exists.
\item[-h] print this help message and exit.
\item[-o \textit{file}] specify an output file. Cannot be used with \texttt{-e} or \texttt{-E}.
\item[-p \textit{prefix}] prefix used in the SQL\TeX\ file. Default is \texttt{sql} (see also section~\ref{config}
on page~\pageref{prefix}. This overwrites the configurarion setting \texttt{\$main::cms\_prefix}.
\item[-q] run in quiet mode.
\item[-s \textit{server}] SQL server to connect to. Default is \texttt{localhost}.
\end{description}


\section{SQL\TeX\ errors and warnings}

\noindent\textbf{\texttt{no input file specified}}

\vspace{1mm}

\noindent SQL\TeX\ was called without any parameters.\\
\textit{Action:} Specify at least one parameter at the commandline. This parameter should be
the name of your input file.

\vs

\noindent\textbf{\texttt{File \textit{input filename} does not exist}}

\vspace{1mm}

\noindent The input file does not exist.\\
\textit{Action:} Make sure the first parameter points to the input file.

\vs

\noindent\textbf{\texttt{outputfile \textit{output filename} already exists}}

\vspace{1mm}

\noindent The outputfile cannot be created because it already exists.\\
\textit{Action:} Specify another output filename with command line option \texttt{-e},
\texttt{-E} or \texttt{-o}, or force an overwrite with option \texttt{-f} (see also section\ref{cmdline}).

\vs

\noindent\textbf{\texttt{no database opened at line \textit{line nr}}}

\vspace{1mm}

\noindent A query starts at line \textit{line nr}, but at that point no database was opened yet. \\
\textit{Action:} Add an \texttt{\bs sqldb} command prior to the first query statement.

\vs

\noindent\textbf{\texttt{insufficient parameters to substitute variable on line \textit{line nr}}}

\vspace{1mm}

\noindent The query starting at line \textit{line nr} uses a parameter in a \textsc{where}- clause with
\texttt{\$PAR\textit{n}}, where \textit{n} is a number bigger than the number of parameters
passed to SQL\TeX\. \\
\textit{Action:} Specify all required parameters at the command line.

\vs

\noindent\textbf{\texttt{trying to substitute with non existing on line \textit{line nr}}}

\vspace{1mm}

\noindent The query starting at line \textit{line nr} requires a variable \texttt{\$VAR\textit{n}} in its
\textsc{where}- clause, where \textit{n} points to a variable that has not (yet) been set. \\
\textit{Action:} Change the number or set the variable prior to this statement.

\vs

\noindent\textbf{\texttt{trying to overwrite an existing variable on line \textit{line nr}}}

\vspace{1mm}

\noindent At line \textit{line nr}, a \texttt{\bs sqlfield} query tries to set a variable \textit{n}
using the option \texttt{[setvar=\textit{n}]}, but \texttt{\$VAR\textit{n}} already
exists at that point. \\
\textit{Action:} Change the number.

\vs

\noindent\textbf{\texttt{no result set found on line \textit{line nr}}}

\vspace{1mm}

\noindent The query starting at line \textit{line nr} returned a \texttt{NULL} value. If the
option \texttt{-N} was specified at the commandline, this is just a warning message.
Otherwise, SQL\TeX\ exits. \\
\textit{Action:} None.

\vs

\noindent\textbf{\texttt{result set too big on line \textit{line nr}}}

\vspace{1mm}

\noindent The query starting at line \textit{line nr}, called with \texttt{\bs sqlfield} returned more than one field. \\
\textit{Action:} Change your query or use \texttt{\bs sqlrow} instead.

\vs

\noindent\textbf{\texttt{unrecognized command on line \textit{line nr}}}

\vspace{1mm}

\noindent At line \textit{line nr}, a command was found that starts with ``\texttt{\bs sql}'',
but this command was not recognized by SQL\TeX\. \\
\textit{Action:} Check for typos. If the command is a user- defined command, it will
conflict with default SQL\TeX\ commands. Change the SQL\TeX\ command prefix (see section~\ref{config}).

\vs

\noindent\textbf{\texttt{no sql statements found in \textit{input filename}}}

\vspace{1mm}

\noindent SQL\TeX\ did not find any valid SQL\TeX\ commands. \\
\textit{Action:} Check your input file.

\section{Copyright and disclaimer}

\noindent\hrulefill \\
The latest release is always available at http://software.oveas.net/sqltex
For bugs, questions and comments, please contact me at
oscar.van.eijk@oveas.net

\vspace{3mm}

\noindent Copyright (c) 2001 - Oscar van Eijk, Oveas Internet Services

\noindent\hrulefill \\


\noindent This software is subject to the terms of the LaTeX Project Public License; 
see http://www.ctan.org/tex-archive/help/Catalogue/licenses.lppl.html.


\end{document}
